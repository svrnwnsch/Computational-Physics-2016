%\documentclass[handout]{beamer}
\documentclass[ignorenonframetext]{beamer}
\usepackage[utf8]{inputenc}
\usepackage[german, english]{babel}
\usepackage{lmodern}
\usepackage[T1]{fontenc}
\usepackage{verbatim}
\usepackage{amsmath}
\usepackage{graphicx}
\usepackage{float}
\usepackage{framed}
\usepackage{mathtools}
\usepackage[font=footnotesize]{caption}
\captionsetup{format=hang}
%\usepackage[hyperfigures =true ,linkcolor =black, urlcolor=blue, colorlinks =true, citecolor=black ,pdfauthor ={ Leonard Peter Wossnig}, ={Analytic solution and Monte Carlo simulation of the two dimensional q-state Potts model},pdfcreator ={ pdfLaTeX }]{hyperref}
%\usepackage{braket}pdftitle
%%%\usepackage[dvips]{color}
%\usepackage[pdftex]{graphicx}
%%%\usepackage{subfigure}
\usetheme{Malmoe}
\useoutertheme{infolines}
%\usepackage{mathpazo}
\parskip1.5ex

\newcommand{\changefont}[3]{\fontfamily{#1}\fontseries{#2}\fontshape{#3}\selectfont}
\definecolor{mygray}{rgb}{0.98,0.98,0.98}
\definecolor{darkgray}{rgb}{0.6,0.6,0.6}
\definecolor{mygreen}{rgb}{0.0,0.5,0.0}
\definecolor{myblue}{rgb}{0.0,0.0,0.5}

\newcommand{\bit}{\begin{itemize}}
\newcommand{\tib}{\end{itemize}}
\newcommand{\kb}{k_\mathrm{B}}
\newcommand{\wc}{\omega_\mathrm{C}}
\newcommand{\q}{q}
\newcommand{\wurzel}{q}
\newcommand{\ci}{\mathrm{i}}



%------------------------------------------------------------------------------------
\title[]{Numerical Solution of Maxwells' Equations in 1D}
\subtitle{}
\author[Wünsch, Yang, Zielke]{\large{Severin Wünsch, Xinrui Yang, Dominikus Zielke} \\
}
\institute[Universität Augsburg]{Institut für Physik der Universität Augsburg} 

\date[29.07.2016]

\titlegraphic{
\hspace{1cm}
%\includegraphics[height=2cm]{ekm}
\hspace{1cm}
%\includegraphics[height=2cm]{unia}
\hspace{1cm}
}


\beamertemplatenavigationsymbolsempty

\setbeamertemplate{headline}
{
  \leavevmode%
  \hbox{%
  \begin{beamercolorbox}[wd=1\paperwidth,ht=2.25ex,dp=1ex,left,leftskip=1em]{section in head/foot}%
    \usebeamerfont{subsection in head/foot}\hspace*{2ex}\insertsectionhead
  \end{beamercolorbox}%
  }%
  \vskip0pt%
}
\makeatletter

\setbeamertemplate{footline}
{
  \leavevmode%
  \hbox{%
  \begin{beamercolorbox}[wd=.5\paperwidth,ht=2.25ex,dp=1ex,center]{author in head/foot}%
    \usebeamerfont{author in head/foot}\insertshortauthor~~\beamer@ifempty{\insertshortinstitute}{}{}
  \end{beamercolorbox}%
  \begin{beamercolorbox}[wd=.5\paperwidth,ht=2.25ex,dp=1ex,right]{date in head/foot}%
    \usebeamerfont{date in head/foot}\insertshortdate{}\hspace*{2em}
  \end{beamercolorbox}}%
  \vskip0pt%
}
\makeatother
\begin{document}
\frame[plain]{\titlepage}

\section*{Table of Contents}
\begin{frame}
\tableofcontents[]
\end{frame}

\section{Problem and approach to a solution}
\begin{frame}
\frametitle{Problem and approach to a solution}
\begin{itemize}
\item Task: Simulate time evolution of Gaussian peak in 1D
\item Idea: Use \textbf{F}inite-\textbf{D}ifference \textbf{T}ime-\textbf{D}omain Method by Yee (1966)
\item Start with Maxwell's Equations in vacuum:
\begin{eqnarray}
\nabla \times \boldsymbol{B}(\boldsymbol{r},t) &=& \frac{1}{c^2}\partial_t{\boldsymbol{E}}(\boldsymbol{r},t)\nonumber\\
\nabla \times \boldsymbol{E}(\boldsymbol{r},t) &=& -\partial_t{\boldsymbol{B}}(\boldsymbol{r}, t)
\end{eqnarray}
\item Further assumptions to simplify the problem:
\begin{itemize}
\item $\boldsymbol{k} \propto \hat{\boldsymbol{z}} \Rightarrow $ E$_z$ = B$_z$ = 0   
\item E$_x$ $\equiv$ E 
\item B$_y$ $\equiv$ B
\end{itemize}
\end{itemize}
\end{frame}

\begin{frame}
\frametitle{Problem and approach to a solution}
\begin{itemize}
\item Evaluating the curls in cartesian coordinates yields
\begin{eqnarray}
-\partial_z B(z, t) &=& \frac{1}{c^2} \partial_t E(z, t)\nonumber\\
\partial_z E(z, t)  &=& -\partial_t B(z, t)
\end{eqnarray}
\item Recall lect.4, pt.2: Central difference approximation
\begin{equation}
f'_i = \frac{f_{i+1} - f_{i-1}}{2h} + \mathcal{O}(h^2) 
\end{equation}
\end{itemize}
\end{frame}

\begin{frame}
\frametitle{Problem and approach to a solution}
\begin{itemize}
\item Yee's idea: Use discretized mesh and displace E- and B- nodes by $i + 1/2$ in space and $j + 1/2$ in time 
\begin{eqnarray}
- \frac{B^{j}(i+\frac{1}{2}) - B^{j}(i-\frac{1}{2})}{\Delta z} &=& \frac{1}{c^2} \frac{E^{j+\frac{1}{2}}(i) - E^{j-\frac{1}{2}}(i)}{\Delta t}\nonumber\\
\frac{E^{j+\frac{1}{2}}(i + 1) - E^{j+\frac{1}{2}}(i)}{\Delta z} &=& -\frac{B^{j+1}(i+\frac{1}{2}) - B^{j}(i+\frac{1}{2})}{\Delta t} 
\end{eqnarray}
\begin{figure}
\centering
%\includegraphics[width=0.7\linewidth]{yee_mesh}
\caption{}
\label{fig:yee_mesh}
\end{figure}

\end{itemize}
\end{frame}

\begin{frame}
\frametitle{Problem and approach to a solution}
\begin{itemize}
\item Solving yields \emph{explicit} formulas for $E^{j+\frac{1}{2}}(i)$ and $B^{j+1}(i+\frac{1}{2})$:
\begin{eqnarray}
E^{j+\frac{1}{2}}(i)&=& E^{j-\frac{1}{2}}(i) -  \frac{c^2 \Delta t}{\Delta z} \left[B^j(i+\frac{1}{2}) - B^j(i-\frac{1}{2})\right] \nonumber\\
B^{j+1}(i+\frac{1}{2}) &=& B^{j}(i+\frac{1}{2}) - \frac{\Delta t}{\Delta z}\left[E^{j+\frac{1}{2}}(i+1) - E^{j+\frac{1}{2}}(i)\right]
\end{eqnarray}
\item Absorbing \emph{boundary conditions}: $E^{j+\frac{1}{2}}(0) = E^{j+\frac{1}{2}}(n)$ = 0 for all time nodes $j$
\item \emph{Initial condition}: $E^{0+\frac{1}{2}}(i) = \exp\left(-\frac{(i - \mu)^2}{2 \sigma^2}\right)$ for all space nodes $i$
\item Illustration of the resulting \emph{leap frog algorithm} is done on blackboard 
\end{itemize}
\end{frame}

\section{Results}
\begin{frame}
\frametitle{Results}
Layout
\begin{itemize}
\item $z_{max} = 8$
\item $t_{max} = 100$
\item $c = 0.1$
\item $\Delta_t$ and $\Delta_z$ gest varied to show stability issues
\end{itemize}
\end{frame}





\section{Summary}
\begin{frame}
\frametitle{Summary}
\end{frame}

\frame[plain]{\centering
\begin{huge}
Thank You for Your Attention
\end{huge}
\newline
\newline
\newline
\newline
\newline
\newline
\newline
\newline
Presentation and Code: \url{github.com/svrnwnsch/Computational-Physics-2016}}


\end{document}

% In case 2 column minipage is needed:
\begin{frame}
\frametitle{Problem and approach to a solution}
\begin{columns}[T] % align columns
\begin{column}{.48\textwidth}
Left Part
\end{column}%
\hfill%
\begin{column}{.48\textwidth}
Right Part
\end{column}%
\end{columns}
\end{frame}